\subsection{Circuit Switching vs. Packet Switching}

\paragraph{Circuit Switching} \mbox{}\\
Circuit switching establishes a dedicated communication path between two devices before data transmission begins. 
This path, or circuit, remains open and reserved for the duration of the communication, even if no data is being transmitted. 
This is similar to a dedicated phone line.

\paragraph{Packet Switching} \mbox{}\\
Packet switching, on the other hand, breaks data into small packets that are independently transmitted over the network. E
ach packet contains the destination address and sequence information, allowing them to be reassembled at the receiver's end. 
The network doesn't establish a dedicated path; packets from different sources can share the same physical links, making more efficient use of network resources. 

\paragraph{Example: Phone Call vs. Email} \mbox{}\\
Imagine a phone call (circuit switching) and sending an email (packet switching):
\begin{itemize}
    \item \textit{Email (Packet Switching)} \newline
    When you dial a number, the phone network establishes a dedicated circuit between your phone and the recipient's phone.
    This circuit remains open for the entire duration of the call, regardless of whether you're speaking or silent.
    No one else can use that circuit during your call.

    \item \textit{Phone Call (Circuit Switching)} \newline
    When you send an email, it is broken down into packets.
    Each packet travels independently across the internet, potentially taking different routes.
    The packets may be interspersed with packets from other users sharing the same network infrastructure.
    At the recipient's end, the packets are reassembled to form the original email.
\end{itemize}
